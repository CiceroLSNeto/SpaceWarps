%%%%%%%%%%%%%%%%%%%%%%%%%%%%%%%%%%%%%%%%%%%%%%%%%%%%%%%%%%%%%%%%%%%%%%%%
%
% Space Warps Publication Policy
%
%%%%%%%%%%%%%%%%%%%%%%%%%%%%%%%%%%%%%%%%%%%%%%%%%%%%%%%%%%%%%%%%%%%%%%%%

\documentclass[a4paper]{article}

% Packages:
\usepackage{xspace}
\usepackage{verbatim}
\usepackage{hyperref}

% Macros:
% JOURNALS
\newcommand{\apj}{ApJ}
\newcommand{\apjl}{ApJL}
\newcommand{\apjs}{ApJS}
\newcommand{\mnras}{MNRAS}
\newcommand{\apss}{Ap \& SS}
\newcommand{\aap}{A\&A}
\newcommand{\aj}{AJ}
\newcommand{\prd}{Phys. Rev. D}
\newcommand{\nat}{Nature}
\newcommand{\araa}{ARA\&A}
\newcommand{\jgr}{J. Geophys. Res.}
\newcommand{\pasp}{PASP}

% MISC
\newcommand{\etal}{et~al.~}
\newcommand{\eg}{{\it e.g.\ }}
\newcommand{\ie}{{\it i.e.\ }}
\newcommand{\etc}{{\it etc.\ }}

\newcommand{\be}{\begin{equation}}
\newcommand{\ee}{\end{equation}}
\newcommand{\bea}{\begin{eqnarray}}
\newcommand{\eea}{\end{eqnarray}}


% CROSS-REFERENCING
\def\Sref#1{Section~\ref{#1}\xspace}
\def\Fref#1{Figure~\ref{#1}\xspace}
\def\Tref#1{Table~\ref{#1}\xspace}
\def\Eref#1{Equation~\ref{#1}\xspace}
\def\Aref#1{Appendix~\ref{#1}\xspace}

% UNITS
\newcommand{\kms}{\ifmmode  \,\rm km\,s^{-1} \else $\,\rm km\,s^{-1}  $ \fi }
\newcommand{\kpc}{\ifmmode  {\rm kpc}  \else ${\rm  kpc}$ \fi  }  
\newcommand{\pc}{\ifmmode  {\rm pc}  \else ${\rm pc}$ \fi  }  
\newcommand{\Msun}{\ifmmode {\rm M_{\odot}} \else ${\rm M_{\odot}}$ \fi} 
\newcommand{\Zsun}{\ifmmode {\rm Z_{\odot}} \else ${\rm Z_{\odot}}$ \fi} 
\newcommand{\yr}{\ifmmode yr^{-1} \else $yr^{-1}$ \fi} 
\newcommand{\hMsun}{\ifmmode h^{-1}\,\rm M_{\odot} \else $h^{-1}\,\rm M_{\odot}$ \fi}

% COSMOLOGY
\newcommand{\LCDM}{$\Lambda{\rm CDM}$}
\newcommand{\MS}{Millennium Simulation\xspace}

% LENSING
\def\zd{z_{\rm d}}
\def\zs{z_{\rm s}}
\def\Dd{D_{\rm d}}
\def\Ds{D_{\rm s}}
\def\Dt{D_{\Delta t}}
\def\Dds{D_{\rm ds}}
\def\Sigmacrit{\Sigma_{\rm crit}}
\def\REin{R_{\rm Ein}}
\def\MEin{M_{\rm Ein}}

% SOFTWARE/HARDWARE
\def\sw{{\small\sc Space\,Warps}\xspace}
\def\SW{{\sc Space\,Warps}\xspace}
\def\Talk{{\small\sc Talk}\xspace}
\def\Letters{{\small\sc Letters}\xspace}
\def\Letter{{\small\sc Letter}\xspace}
\def\Dashboard{{\small\sc Dashboard}\xspace}
\def\cfhtls{{\it CFHTLS}\xspace}
\def\python{{\sc python}\xspace}

% TABLES:
\newcommand\nodata{ ~$\cdots$~ }%

% PROBABILITY THEORY
\def\pr{{\rm Pr}}
\def\data{{\mathbf{d}}}
\def\datap{{\mathbf{d}^{\rm p}}}
\def\datai{d_i}
\def\datapi{d^{\rm p}_i}
\def\LENS{{\rm LENS}}
\def\saidLENS{{\rm ``LENS"}}
\def\NOT{{\rm NOT}}
\def\saidNOT{{\rm ``NOT"}}

% AGENT BUREAUCRACY
\def\effort{N_{\rm C}}
\def\experience{N_{\rm T}}
\def\skill{\langle I \rangle}
\def\contribution{$\sum_k \skill_k$}
\def\information{\delta I}

% COMMENTING
\usepackage[usenames]{color}
\newcommand{\question}[2]{\textcolor{red}{\bf Question from #1: #2}}
\newcommand{\flag}[2]{\textcolor{blue}{\bf Comment from #1: #2}}
\newcommand{\new}[1]{{\bf #1}}

% RESULTS
\def\Ncollaboration{XXX}


%%%%%%%%%%%%%%%%%%%%%%%%%%%%%%%%%%%%%%%%%%%%%%%%%%%%%%%%%%%%%%%%%%%%%%%%

\begin{document}
             
\title{\SW Collaboration Publication Policy: CFHT-LS}
\author{Phil Marshall \& Aprajita Verma}
\date{April 15, 2013}
\maketitle

%%%%%%%%%%%%%%%%%%%%%%%%%%%%%%%%%%%%%%%%%%%%%%%%%%%%%%%%%%%%%%%%%%%%%%%%%%%%%%

\begin{abstract} 
\noindent \sw was conceived as a strong lens
discovery {\it service}, that any survey science team can make use of
(within the logistical constraints). 
In this document, we define 
the various types of \sw
collaboration membership, and their meaning in terms of the \sw
paper series. This includes guidelines for publications of \sw-enabled discoveries. We also discuss the \sw data usage policy and outline the first set of \sw papers.

As \sw was developed based on the requirements of the first \sw project, CFHT-LS, the policy for the first project differs from subsequent \sw projects involving new data. This document is written specifically for the CFHT-LS project and subsequent projects should refer to the ``Guidelines and Policy for Surveys using \sw'' document.
\end{abstract}

\setcounter{footnote}{0}

%%%%%%%%%%%%%%%%%%%%%%%%%%%%%%%%%%%%%%%%%%%%%%%%%%%%%%%%%%%%%%%%%%%%%%%%%%%%%%


\section{Collaboration Membership Types}
\label{sec:members}

The \sw website is designed to motivate and  enable tens of thousands of
people to perform the scientific tasks of strong gravitational lens
identification and classification.  The \sw collaboration comprises the
sum total of the volunteers who have logged in and contributed
classifications to the project (we would include those who did not
log in as well, but we don't know who they are!). This means that the
minimal requirement for \sw collaboration membership is logging in with a Zooniverse ID
and providing some contribution on the site (\ie classifying, or discussing
Subjects in \Talk \footnote{\Talk (found at talk.spacewarps.org) is a forum for \sw collaborators to discuss individual sources and the \sw project in general. Bugs can also be reported. This is the direct portal for communication with the \sw community.}).\\

The collaboration membership is broadly split into four subsets:

\begin{description}

\item{\bf Principal Investigators:} These are the PIs of the \sw project (P.\,Marshall, 
A.\,More and A.\,Verma)\footnote{The PIs may be collectively contacted via the email address spacewarpspi@googlegroups.com}

\item{\bf Zooniverse:} \sw development team (originally led by
A.\,Kapadia).

\item{\bf Science Team:} This is a core group of scientists, both citizen and
professional, that helped to design, build and test the \sw website
and the initial CFHT-LS project. The Science Team includes the \sw PIs.

\item{\bf Community} All of the volunteers who have contributed
classifications to the project\footnote{We can only communicate directly with those who have logged in with a Zooniverse ID via \Talk or group e-mail.}.

\item{\bf Affiliated:} This group includes professional scientists who have made some contribution to the \sw project, but smaller than the Science Team. Such contributions might include helping write the initial
Citizen Science Alliance proposal, participating in discussions on
\Talk, writing \sw blog posts, advising on the data analysis, or editing
papers.

\end{description}



%First, however, we remind ourselves about data access via \sw.

%%%%%%%%%%%%%%%%%%%%%%%%%%%%%%%%%%%%%%%%%%%%%%%%%%%%%%%%%%%%%%%%%%%%%%%%%%%%%%

%\section{Survey Data Use}
%\label{sec:data}

%\sw is a public website. All images displayed there are by definition in the
%public domain, and so must be expected to be downloaded, copied and
%redistributed by any \sw user. This is a \textbf{good thing}: images of the sky taken
%with publicly-funded telescopes belong to everyone. It is the responsibility
%of the survey science teams to provide images in a way that is consistent
%with their own data access rules. 

%While the CFHT-LS is a public survey, the WCS information has been removed from the provided images. This prevents the follow-up of any of the objects in the survey by anyone other than the \sw
%team. In practice, the objects contained in the \sw images will be too faint
%for anyone without access to a large telescope to observe, and may also be
%absent from any public catalogs, so the opportunities for follow-up will be
%quite limited. Removing the WCS information is an insurance policy against any
%competing professional scientists looking to ``scoop'' the survey team,
%despite the damage it would do to their reputations.\footnote{We cannot exclude the possibility that object co-ordinates are posted in \Talk. For example, co-ordinates of previously published lens candidates or recognisable fields from CFHT-LS have appeared in \Talk.} 



%%%%%%%%%%%%%%%%%%%%%%%%%%%%%%%%%%%%%%%%%%%%%%%%%%%%%%%%%%%%%%%%%%%%%%%%%%%%%%

\section{The \SW Papers}

In the following sections we look at the various publications that might
arise from the \sw project, and suggest reasonable guidelines for
deciding on their authorship. 

\subsection{The \sw System Paper Series}
\label{sec:series}

We are planning a short series of journal papers describing the \sw service,
its first dataset (CFHT-LS), and results from its investigation by the
collaboration. These results will likely include a sample of new lenses, and
some comparison studies between the visual and automated identification of
lens systems in this survey. This series will be known as ``the \sw papers,''
and each one will have a title that starts with ``\sw.''

The initial papers we are planning to write include (in no particular order,
and the actual paper series may be split or change in content and order):

\begin{itemize}
% Old versions of this list, which caused a conflict when I (PJM) merged in
% Aprajita's May 2014 edits. Both look out of date to me! Suggested new 
% version is beneath these.
% 
% <<<<<<< HEAD
% \item \sw I: The system paper, describing the project, the interface, the classification analysis methods (SWAP) and results 
% \item \sw II: The \sw  first results including simulations and community performance (not selection function), means by which you can compute a selection function for any given analysis, list of candidates. 
% \item \sw III: Detailed results on candidates from CFHT-LS \textit{including models, follow-up etc}
% \item \sw IV: Detailed comparison and analysis of citizen scientist classification versus lens-finding
% \item \sw X: New lenses(s) in \sw (authored by collaboration members)
%
% =======
% \item \sw I: The system paper, describing the project, the interface, the
% classification analysis methods (SWAP) and results 
% \item \sw II: The \sw  simulations and community performance (not selection
% function), means by which you can compute a selection function for any given
% analysis. 
% \item \sw III: The first new candidates from CFHT-LS \textit{including models,
% follow-up if feasible}
% \item \sw IV: The results from CFHT-LS search including the comparison and
% analysis of citizen scientist classification versus lens-finding algorithms
% (maybe split into smaller papers).
% >>>>>>> 34d2fe73486cc68b301665b1b811cff7804137e8
%
\item \sw I: The system paper, describing the project, the interface \& training scheme, the classification analysis methods (SWAP) and results from classification of the training set. (Marshall et al.)
\item \sw II: The \sw CFHTLS project, including description of CFHT-LS specific information (e.g. data/subjects, generation of simulated images), results new lens candidates, and preliminary analysis of \sw compared to previous semi-automatic lens searches in this survey. (More et al.)
\item \sw III: Detailed comparison and analysis of citizen scientist classification versus lens-finding. (More et al.)
%AV to PJM: i thought Anu still wanted the sim-automated serach comparison to be considered in a separate paper
\item \sw IV-X: Detailed results on candidates from CFHT-LS \textit{including lens models, follow-up etc.} (IV: Verma et al. \& authored by collaboration members)
\end{itemize}

There maybe further papers written in this series. \newline

\subsection{\sw Related Papers}
These are \sw related papers but do not necessarily fall into the \sw paper series. Planned examples are:
\begin{itemize}
\item Lens modelling in \sw with {\sc Spaghetti Lens})
\item New lenses(s) in \sw (authored by persons not part of the \sw collaboration)
\end{itemize}


Please email the \sw PIs at \href{mailto:spacewarpspi@googlegroups.com}{spacewarpspi@googlegroups.com} if you have any
ideas for further \sw series or related papers. 
 
%%%%%%%%%%%%%%%%%%%%%%%%%%%%%%%%%%%%%%%%%%%%%%%%%%%%%%%%%%%%%%%%%%%%%%%%%%%%%%

\section{Publication Policy}
\label{sec:publ}


The authorship policy for \sw series and related papers is defined here.


\subsection{Journal Papers}
\label{sec:series}

A core team of collaborators will participate in writing journal papers, and
many collaborators who have made a \textit{significant contribution} to \sw
will be recognised on the author lists of journal papers presenting \sw
results.  However, not all collaboration members will participate in writing
journal papers.  \textit{Significant contributions} include those who have
contributed to:
\begin{itemize}
\item the site design, construction and operation
\item data preparation or analysis
\item writing and editing the papers. 
\end{itemize}


The current list of \sw authors is given in the \texttt{authors.tex} file and is reprinted
in the appendix to this document (see Appendix \ref{sec:appa}); this file shown is as it is currently configured, ready to be copied and pasted into the \sw system paper (\sw I). 

\noindent The author categories are grouped according to collaboration membership (defined above). These authors have either the status to ``opt-out'' or "opt-in" on \sw publications.  \\
``Opt-out'' means the listed people are authors by default on \sw series papers and must explicitly state if they wish for their names to be removed from \sw publications. \\
``Opt-in'' means the people listed should be invited to comment on and contribute to publications however it is the decision of the lead author whether to add their names based on what they have contributed.

For the CFHT-LS \sw system papers, the Principal Investigators, Zooniverse development team and \sw Science team all have ``opt-out'' status and the \sw Affiliates have ``opt-in'' status.

The first author of any \sw paper will decide, as usual, on the author list of their own paper. This will include adding Affiliates who they think have made a substantial contribution to that investigation, as well as inviting scientists from outside the list of \sw authors, or from outside the
collaboration.  The \sw PIs will act as adjudicators if any disputes arise, but to be honest we are not expecting any.

Lead authors of \sw papers should circulate drafts to the list of \sw authors (according to the policy above) and allow a minimum of two weeks to give time for comments. If  ``opt-in'' authors have not provided comments within this time frame they will *not* be included as authors on the paper. If ``opt-out'' authors have not responded to ask for their names to be removed they will be kept on the author list.



For \sw related papers, the \sw PIs should be contacted via \href{mailto:spacewarpspi@googlegroups.com}{spacewarpspi@googlegroups.com} for a suggested minimum contingent of collaboration authors to be added to the paper, consistent with the policy for general/future \sw projects (please see Guidelines and Policy for Surveys using \sw in this directory). 


%%%%%%%%%%%%%%%%%%%%%%%%%%%%%%%%%%%%%%%%%%%%%%%%%%%%%%%%%%%%%%%%%%%%%%%%%%%%%%

\subsection{Zooniverse Publications: Letters and Talk }
\label{sec:zoolet}


Any \sw collaboration member, simply by virtue of their Zooniverse
registration, may write a Zooniverse
Letter\footnote{\texttt{http://letters.zooniverse.org}} describing their
investigation of any lens candidate they find in \sw. This, along with
posts in  \Talk, are the primary means by which we expect collaboration
members will communicate their findings to the rest of the
astronomical community. 
The \sw Zooniverse \Letters will contribute to the collective knowledge and understanding of the presented system(s). 
Zooniverse \Letters are citable
objects, and in some cases will appear listed on ADS.\footnote(To be confirmed.)
We strongly encourage community members to write Zooniverse \Letters about their findings. We would be happy to help any member wishing to do this, please just drop us a line at spacewarpspi@googlegroups.com. 


As \sw classifications are a community wide activity, it is impossible to attribute the discovery of a candidate to a single community member, or group  of members. Therefore, in recognition of their contribution to the \sw project, all \sw community members will be listed (on their approval) on the \sw members web page (URL: to be added). This web page will be linked to in the acknowledgements of every ensuing \sw publication. If a community member makes a significant contribution e.g.\,in the further investigation of a lens candidate with modelling tools, they are strongly encouraged to write Zooniverse Letters (see Section \ref{sec:zoolet}). This may lead to them being invited to join journal publications by the lead author.  

%%%%%%%%%%%%%%%%%%%%%%%%%%%%%%%%%%%%%%%%%%%%%%%%%%%%%%%%%%%%%%%%%%%%%%%%%%%%


\section{Publication Guidelines}
\label{sec:publ}

Any \sw series and \sw enabled discovery papers should include
\begin{itemize}
\item citations to the \sw system papers, \sw I \& II
\item the collaboration should be acknowledged in the acknowledgement section with a link to the collaboration membership page (URL: to be added).
\item \sw CFHT-LS is based on data processed by CFHT-LS and Terapix, so please add the following acknowledgement to your paper:\\
\textit{``Based on observations obtained with MegaPrime/MegaCam, a joint project of CFHT and CEA/IRFU, at the Canada-France-Hawaii Telescope (CFHT) which is operated by the National Research Council (NRC) of Canada, the Institut National des Science de l'Univers of the Centre National de la Recherche Scientifique (CNRS) of France, and the University of Hawaii. This work is based in part on data products produced at Terapix available at the Canadian Astronomy Data Centre as part of the Canada-France-Hawaii Telescope Legacy Survey, a collaborative project of NRC and CNRS.''}\\
from http://www.cfht.hawaii.edu/Science/CFHLS/cfhtlspublitext.html

\end{itemize}

%%%%%%%%%%%%%%%%%%%%%%%%%%%%%%%%%%%%%%%%%%%%%%%%%%%%%%%%%%%%%%%%%%%%%%%%%%%%

\appendix
\section{Current Author List for \sw series papers:}
\label{sec:appa}
{\small \verbatiminput{authors.tex}}

%%%%%%%%%%%%%%%%%%%%%%%%%%%%%%%%%%%%%%%%%%%%%%%%%%%%%%%%%%%%%%%%%%%%%%%%%%%%%%

\end{document}

%%%%%%%%%%%%%%%%%%%%%%%%%%%%%%%%%%%%%%%%%%%%%%%%%%%%%%%%%%%%%%%%%%%%%%%%%%%%%%

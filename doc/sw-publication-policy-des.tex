%%%%%%%%%%%%%%%%%%%%%%%%%%%%%%%%%%%%%%%%%%%%%%%%%%%%%%%%%%%%%%%%%%%%%%%%
%
% Space Warps Guidelines & Policy
%
%%%%%%%%%%%%%%%%%%%%%%%%%%%%%%%%%%%%%%%%%%%%%%%%%%%%%%%%%%%%%%%%%%%%%%%%

\documentclass[a4paper,twocolumn]{article}

% Packages:
\usepackage{xspace}
\usepackage{verbatim}
\usepackage{hyperref}

% Macros:
% JOURNALS
\newcommand{\apj}{ApJ}
\newcommand{\apjl}{ApJL}
\newcommand{\apjs}{ApJS}
\newcommand{\mnras}{MNRAS}
\newcommand{\apss}{Ap \& SS}
\newcommand{\aap}{A\&A}
\newcommand{\aj}{AJ}
\newcommand{\prd}{Phys. Rev. D}
\newcommand{\nat}{Nature}
\newcommand{\araa}{ARA\&A}
\newcommand{\jgr}{J. Geophys. Res.}
\newcommand{\pasp}{PASP}

% MISC
\newcommand{\etal}{et~al.~}
\newcommand{\eg}{{\it e.g.\ }}
\newcommand{\ie}{{\it i.e.\ }}
\newcommand{\etc}{{\it etc.\ }}

\newcommand{\be}{\begin{equation}}
\newcommand{\ee}{\end{equation}}
\newcommand{\bea}{\begin{eqnarray}}
\newcommand{\eea}{\end{eqnarray}}


% CROSS-REFERENCING
\def\Sref#1{Section~\ref{#1}\xspace}
\def\Fref#1{Figure~\ref{#1}\xspace}
\def\Tref#1{Table~\ref{#1}\xspace}
\def\Eref#1{Equation~\ref{#1}\xspace}
\def\Aref#1{Appendix~\ref{#1}\xspace}

% UNITS
\newcommand{\kms}{\ifmmode  \,\rm km\,s^{-1} \else $\,\rm km\,s^{-1}  $ \fi }
\newcommand{\kpc}{\ifmmode  {\rm kpc}  \else ${\rm  kpc}$ \fi  }  
\newcommand{\pc}{\ifmmode  {\rm pc}  \else ${\rm pc}$ \fi  }  
\newcommand{\Msun}{\ifmmode {\rm M_{\odot}} \else ${\rm M_{\odot}}$ \fi} 
\newcommand{\Zsun}{\ifmmode {\rm Z_{\odot}} \else ${\rm Z_{\odot}}$ \fi} 
\newcommand{\yr}{\ifmmode yr^{-1} \else $yr^{-1}$ \fi} 
\newcommand{\hMsun}{\ifmmode h^{-1}\,\rm M_{\odot} \else $h^{-1}\,\rm M_{\odot}$ \fi}

% COSMOLOGY
\newcommand{\LCDM}{$\Lambda{\rm CDM}$}
\newcommand{\MS}{Millennium Simulation\xspace}

% LENSING
\def\zd{z_{\rm d}}
\def\zs{z_{\rm s}}
\def\Dd{D_{\rm d}}
\def\Ds{D_{\rm s}}
\def\Dt{D_{\Delta t}}
\def\Dds{D_{\rm ds}}
\def\Sigmacrit{\Sigma_{\rm crit}}
\def\REin{R_{\rm Ein}}
\def\MEin{M_{\rm Ein}}

% SOFTWARE/HARDWARE
\def\sw{{\small\sc Space\,Warps}\xspace}
\def\SW{{\sc Space\,Warps}\xspace}
\def\Talk{{\small\sc Talk}\xspace}
\def\Letters{{\small\sc Letters}\xspace}
\def\Letter{{\small\sc Letter}\xspace}
\def\Dashboard{{\small\sc Dashboard}\xspace}
\def\cfhtls{{\it CFHTLS}\xspace}
\def\python{{\sc python}\xspace}

% TABLES:
\newcommand\nodata{ ~$\cdots$~ }%

% PROBABILITY THEORY
\def\pr{{\rm Pr}}
\def\data{{\mathbf{d}}}
\def\datap{{\mathbf{d}^{\rm p}}}
\def\datai{d_i}
\def\datapi{d^{\rm p}_i}
\def\LENS{{\rm LENS}}
\def\saidLENS{{\rm ``LENS"}}
\def\NOT{{\rm NOT}}
\def\saidNOT{{\rm ``NOT"}}

% AGENT BUREAUCRACY
\def\effort{N_{\rm C}}
\def\experience{N_{\rm T}}
\def\skill{\langle I \rangle}
\def\contribution{$\sum_k \skill_k$}
\def\information{\delta I}

% COMMENTING
\usepackage[usenames]{color}
\newcommand{\question}[2]{\textcolor{red}{\bf Question from #1: #2}}
\newcommand{\flag}[2]{\textcolor{blue}{\bf Comment from #1: #2}}
\newcommand{\new}[1]{{\bf #1}}

% RESULTS
\def\Ncollaboration{XXX}


%%%%%%%%%%%%%%%%%%%%%%%%%%%%%%%%%%%%%%%%%%%%%%%%%%%%%%%%%%%%%%%%%%%%%%%%

\begin{document}
             
\title{Guidelines and Policy for the Dark Energy Survey Strong Lensing Science Team using \SW}
\author{Aprajita Verma, Phil Marshall \& Anupreeta More}
\date{March 25, 2015}
\maketitle

%%%%%%%%%%%%%%%%%%%%%%%%%%%%%%%%%%%%%%%%%%%%%%%%%%%%%%%%%%%%%%%%%%%%%%%%%%%%%%

\begin{abstract} 
\noindent \sw was conceived as a strong lens
discovery {\it service} by harnessing the power of citizen scientists. The the DES strong lensing group (hereafter DES survey science team or DES SST) can make use of
(within the logistical constraints). In this document we discuss how the DES SST can become involved with \sw and the relationship
between the DES survey science team and the \sw collaboration. 
This document includes guidelines for proposing projects, data provision, data policy and for publications of \sw-enabled discoveries.
\end{abstract}

\setcounter{footnote}{0}

%%%%%%%%%%%%%%%%%%%%%%%%%%%%%%%%%%%%%%%%%%%%%%%%%%%%%%%%%%%%%%%%%%%%%%%%%%%%%%

\section{Introduction to \sw}

The \sw website is designed to motivate and  enable tens of thousands of
people to perform the scientific tasks of strong gravitational lens
identification and classification.  \sw comprises a primary classification interface (http://spacewarps.org) that allows citizen scientists to inspect images and mark those that contain potential gravitational lenses. In addition, these volunteers are shown simulated lenses (appropriate for the survey being conducted) as an interactive training tool that also allows us to establish the citizens' likelihood of identifying potential lenses and blank fields using the \sw Analysis Pipeline (SWAPR). 

A key part of the \sw project is the ability for all members of the \sw collaboration (see Section \ref{sec:members}) to communicate with each other and discuss interesting candidates through \Talk\footnote{\Talk (found at \texttt{http://talk.spacewarps.org}) is a forum for \sw collaborators to discuss individual sources and the \sw project in general. Bugs can also be reported. This is the direct portal for communication with the \sw community.}).

In order to use \sw interface, the DES Survey Science Team \textbf{must define and propose a project to the \sw PIs} who will establish whether the use of the \sw interface is appropriate and liaise with the DES Survey Science Team to implement their project in the \sw interface. Details are given in  Section \ref{sec:project}.

\vspace{2mm}
The PIs of the \sw project are the authors of this document, P.\,Marshall, 
A.\,More and A.\,Verma, please do not hesitate to contact us if you require any further information or have any questions. The PIs may be collectively contacted via the email address \href{mailto:spacewarpspi@googlegroups.com}{spacewarpspi@googlegroups.com}


%%%%%%%%%%%%%%%%%%%%%%%%%%%%%%%%%%%%%%%%%%%%%%%%%%%%%%%%%%%%%%%%%%%%%%%%%%%%%%


\section{Definition of collaboration member types}
\label{sec:members}

In this document, we will refer to the entirety of the DES strong lensing science group and the STRIDES team as the DES Survey Science Team (DES SST) to encompass all of the DES strong lensing community.

We now discuss the relationship between the \sw collaboration and the DES SST, who wish to use the \sw interface to investigate their data. The DES SSTs will liaise with the \sw science and development teams to ingest their survey data into the \sw interface. There may be some overlap between current \sw members and the future SSTs. In the case of DES, \sw PI Anupreeta More is a member of both the \sw collaboration and the DES Strong Lensing Science Group.

When we refer to the \sw collaboration, it comprises the \sw science team, \sw development team, and all the volunteers who have 
logged in\footnote{with a Zooniverse ID. It is possible to classify anonymously, and we count those classification but as we don't know who they are, we are unable to collaborate with them directly.} and contributed classifications to the project. 

If required, an expanded definition of the \sw membership is given in the document ``\sw Publication Policy: CFHT-LS''., however this will not be relevant for publications, see Section \ref{sec:papers} for details.

%%%%%%%%%%%%%%%%%%%%%%%%%%%%%%%%%%%%%%%%%%%%%%%%%%%%%%%%%%%%%%%%%%%%%%%%%%%%%%


\section{Defining a suitable \sw science project}
\label{sec:project}

The DES SST is expected to propose a project to the \sw collaboration that can be best or only be done using the power of citizen scientists. Typically this involves visual inspection tasks that are too time consuming for the SST themselves to execute. As such any science programme conducted through \sw should be complementary, not in competition, to the lens finding activities being carried out by the SST.

For example, in the first \sw project, the \sw collaboration conducted a lens finding search with \sw over the entire CFHT-LS survey area. The goal was to increase the completeness of strong lenses in the survey areas. Both RingFinder and ArcFinder algorithms had been used to discover lenses of different types. \sw complemented these searches with the potential to find strong lenses missed by these algorithms. The method, proposed by Anupreeta More (who served as the SST for this project), was to inspect the entire survey area by getting citizens to inspect $\sim$430,000 tiles of the survey area. This was a \textit{blind} search. Such a search could not be conducted by a normal sized SST.

\textit{Targeted} searches are also possible, as we performed in the second \sw project on the VICS82 survey area. Here we inspected $\sim$40,000 images from the VICS82 survey centred on likely strong lens targets, e.g. LRGs, groups, clusters, quasars.  Again this was more images than the SST could manage to inspect alone.

Another approach is to use the \sw service to inspect the typically large output from automated algorithms or searches that are typically dominated by false positives.

\subsection{DES specific thoughts}
In the case of DES, with its long data release plan and large strong lens group, we request the team consider the following:
\begin{enumerate}
\item The \sw facility should be used for science projects that cannot be completed by the Strong Lens Group itself. 
\item Any \sw search should be complementary, not in competition, to the lens finding being conducted by the DES SST itself.
\item While \sw uses citizen science, its primary goal is \textbf{not outreach}, it is harnessing manual power for visual classifications (and eventually modelling) power by citizens to perform science.
\item We should be careful not to propose a duplicate project provided there is a reasonable scientific motivation behind to do so 
\item \sw requires volunteers/citizens to give up their personal time to conduct the experiment, this should not be taken lightly, the crowd are motivated by contributing to real science not just doing things for the sake of them. This is the reason the suitability of the project for citizen scientists should be carefully considered. The \sw PIs will be able to discuss any planned \sw-DES projects with the DES SST.
\item The success of \sw-DES will rely on input from the DES SST working with the \sw collaboration, dedicated effort is required for this and the DES SST should identify resource for this before embarking on a project. The \sw PIs will help to establish the main efforts required that depend on the actual project put forward by the SST (i.e. image preparation, simulations).
\end{enumerate}

\textbf{As such, \sw should be considered to be an integral part of the DES SST lens finding strategy, rather than an alternative method.}

The scientific project, process and outcome is the domain of the SST, the \sw collaboration enables this work by making it a live project on the website. We do not, per se, expect the \sw collaboration lead the resultant papers, we just request some of us are included on discovery papers in return for our help, see Section \ref{sec:papers} for more details on the publication policy.

Please do remember that \sw is setup to provide a service to help your team find lenses, any discovery in \sw-DES is the DES SSTs discovery. The \sw PIs or the \sw collaboration are not interested in scooping your targets. Using SWAP we will send you lists of the candidates selected by the citizens (when you require or at the end of a project). If you wish to have our input or collaboratively work on targets we are happy to become more involved (the \sw Pis have strong lensing science interests) but we will not do anything with the sources found through \sw other than provide it to the DES SST.  We are happy to then work with you as you wish. 

Note all possible candidates are in the public domain as all images are available online (but without WCS information). These targets may be discussed by citizens on \sw TALK that the DES SST are expected to engage in discussions with citizens.

\subsection{Previewing images}
We are aware that there is some interest in the DES SST for using \sw to preview images for the SST themselves. As this is not the function of the \sw interface or the Zooniverse, this woud only be possible under certain conditions
(a) the images inspected are destined to be seen by the crowd either in parallel with the SST or immediately following the SST inspection
(b) the framework of the image presentation (i.e the \sw interface) is the same as the interface presented to the user (image clicks only and no comments or ranking possible)
(c) all SST inspectors would have to register with the Zooniverse to classify
(d) While we would not distribute the URL to community, we cannot password protect the site. Note the data can still be stripped of WCS information (please see Section \ref{sec:data} for more information.)

Anupreeta More's \textit{visapp} might be a more suitable tool for the SST to use.


\subsection{Possible \sw-DES survey strategies}

We strongly recommend that the majority of the DES SST are behind a search for strong lenses with \sw so we would like that the actual \sw project is discussed with all, highlighting where \sw can complement ongoing or planned searches by teams within the DES SST. We are here to assist your science! While we can't expect the entirety of the DES Strong Lensing Group or the STRIDES consortium to be behind \sw-DES, all should be made aware of the projects being undertaken and planned with \sw.

In particular for the first potential \sw-DES project (probably on the Y1 data) we encourage the DES SST to strongly consider the timing for when they wish to ingest the images into \sw. This could be delayed until the SST has performed all the visual inspection or automated algorithms they wish to to avoid any issue of competition with \sw (if that is a concern).

Possible projects for the DES SST could consider (just to get a flavour of what can be done)
\begin{enumerate}
\item \textbf{New Data}:Inspection of areas (targeted or blind) that the SST has not had time to inspect, i.e. use the crowd to increase the speed at which potential lens candidates are discovered
\item \textbf{Comparison of Expert vs Citizen Classification}: DES SST and citizen's view images in parallel (targeted or blind)
\item \textbf{Completeness of Expert Classification}: Citizen's classify images already inspected by the DES SST (targeted or blind)
\item \textbf{Helping the robots}: Use citizens to weed out false positives from candidate samples generated by algorithm based searches
\item other...
\end{enumerate}

Note that the projects that you propose to \sw need not be one of the above. You can also suggest multiple projects that we could run in a mixture or phased set of experiments. Please do consider your timeline with reference to data release and the searches the DES SST want to make themselves.

If the DES SST wants to inspect targets within the consortium and is concerned about being scooped, we suggest that this should happen first before images are passed to \sw. We then label any images containing lenses already discovered by the SST as `known lenses'. This applies to visually inspected by the SST or algorithm searched images. 


\subsection{What to do next: Project Proposals}
Once you have narrowed down what search you would like to do with \sw, please approach the \sw PIs (Phil Marshall, Anupreeta More, Aprajita Verma, collectively contacted via \href{mailto:spacewarpspi@googlegroups.com}{spacewarpspi@googlegroups.com}) with a short description of their project that includes the following information.

\begin{itemize}
\item Survey area
\item Bands to be included
\item Total number of images to be viewed (approx.)
\item Targeted survey or blind field search \footnote{Targeted: inspection of a list of candidates; Blind: inspection of tiles of a survey area}
\item Brief outline of the overall goal(s) of that inspection, highlighting how this is complementary (not competing) with the goals of the DES lens finding group
\item Proprietary data issues 
\end{itemize}

Note, the proposed science project(s) should harness citizen classification power to achieve your science goals.

The \sw PIs will then arrange a telecon with the SSTs to discuss their project further.

The original \sw papers, by definition, provide a complete description of the system, and its results from the CFHT-LS survey. The SST will be able
to read these papers before designing their own \sw project, and cite them as justification for some of their experimental decisions. As the \sw paper series get published, links to the papers will be given in this draft. Prior to publications, draft papers can also be shared on request by the \sw PIs (\href{mailto:spacewarpspi@googlegroups.com}{spacewarpspi@googlegroups.com})


In the remaining sections we look at discovery publications that might
arise from future \sw projects, and suggest reasonable guidelines for
deciding on their authorship. First, however, we remind ourselves about data
access via \sw.

%%%%%%%%%%%%%%%%%%%%%%%%%%%%%%%%%%%%%%%%%%%%%%%%%%%%%%%%%%%%%%%%%%%%%%%%%%%%%%


\section{Survey Data Policy}
\label{sec:data}

\sw is a public website. All images displayed there are by definition in the
public domain, and so must be expected to be downloaded, copied and
redistributed by any \sw user. This is a \textbf{good thing}: images of the sky taken
with publicly-funded telescopes belong to everyone. 

However, some surveys including DES come with their own proprietary access policy. It is the responsibility
of the DES SST to provide images to \sw in a way that is consistent
with their own data access rules. There are two things that SSTs can
do to in order to respect any proprietary period that they have imposed on
themselves:
\begin{enumerate}

\item Add a ``LICENSE'' keyword to the FITS and PNG image headers, explaining
what the rules for redistribution of these images are. This will almost
certainly be ignored, but it would be a nice reminder that nothing comes free
of either cost or responsibility. Other keywords could also be included as
well: data provenance is important, and links to useful survey webpages would
be most welcome!

\item Remove all WCS information from all images provided. The \sw interface ``dashboards'' do allow FITS and JPG files to be downloaded. If images from a proprietary dataset being made available is a concern for an SST, they should remove the WCS. This also has the consequence that it minimises the potential ``scooping'' of  lens candidates by anyone other than the SST. 
In practice, the objects contained in the \sw images will be too faint
for anyone without access to a large telescope to observe, and may also be
absent from any public catalogs, so the opportunities for follow-up will be
quite limited. Images
with field of view less than about 3~arcminutes in diameter are not solvable
by \texttt{astrometry.net}. \textbf{Please note}, however, that we cannot exclude the possibility that object co-ordinates are posted in \Talk. For example, co-ordinates of previously published lens candidates or recognisable fields from CFHT-LS have been posted in \Talk by a very small number of citizens.

\end{enumerate}

%%%%%%%%%%%%%%%%%%%%%%%%%%%%%%%%%%%%%%%%%%%%%%%%%%%%%%%%%%%%%%%%%%%%%%%%%%%%%%

\section{Data Preparation}
\label{sec:data}

\subsection{Collaborating with \sw Science Team}

Implementing a new project based on different survey data will require support from the \sw Science Team, in three respects: 

\begin{enumerate}

\item In {\bf preparing data} for the site, and uploading it onto the
Zooniverse servers. There is a reasonable amount of work in preparing data and images by the DES SST for delivery to \sw. The \sw team will provide assistance to the DES SST in data preparation. The benefits of using the \sw service will far outweigh the time needed to prepare data.  The \sw Science Team can advise on suitable Training
Subjects and image display settings, and also on formatting the data
ready for ingestion and display.

\item In {\bf reconfiguring the site itself} ready for the new data. The DES
SST will need to include new Spotter's Guide images and text,
modified tutorial content, and additional survey-specific site content;
the \sw Science Team can help with all of this.

\item In {\bf maintaining the site} so that it continues to function
correctly. This will involve technical support from the Zooniverse and \sw science teams,
but will also require the continued engagement of the wider
collaboration by the SST in \Talk. Since this social filter is
a key part of the sample refinement, the latter may involve a
significant time commitment.

\end{enumerate}



%%%%%%%%%%%%%%%%%%%%%%%%%%%%%%%%%%%%%%%%%%%%%%%%%%%%%%%%%%%%%%%%%%%%%%%%%%%%%%

\section{Resulting publications for \sw-DES }
\label{sec:series}

\subsection{Journal Papers}
\label{sec:papers}
In return for the work outlined in Section \ref{sec:series} in setting up the new surveys, the \sw PIs will nominate a
small number of core \sw collaboration members as authors on \textbf{papers that
present discovery of a lens or sample of lenses enabled by the \sw system}. A ``discovery'' is defined by the announcement of the source(s) to the astronomical community in a journal paper. We would expect the \sw author list to be a small number (less than ten), primarily the Principal Investigators and a few others working on the ingestion of the new data. All post-discovery follow-up papers would of course be exempt from this: the idea is that \sw is purely a gravitational lens discovery service, but not one that runs by magic.

We request that authors of \sw-enabled discovery papers should circulate drafts to the list of \sw authors and allow a minimum of two weeks to give time for comments. We also request that any papers based on \sw enabled discoveries should cite the \sw system papers. This policy document will be updated with the correct citations to the \sw system papers when available.

\sw runs an analysis pipeline (SWAPR) that will be tailored to the science goals of each new survey. Any large changes to the analysis code will be documented  by the \sw Science Team. The \sw Science Team reserve the right to update the \sw methodology papers.



\subsection{Zooniverse \Letters}
\label{sec:comm}

Any \sw collaboration member, simply by virtue of their Zooniverse
registration, may write a Zooniverse
Letter\footnote{\texttt{http://letters.zooniverse.org}} describing their
investigation of any lens candidate they find in \sw. This, along with
posts in  \Talk, is the primary means by which we expect collaboration
members will communicate their findings to the rest of the
astronomical community. The investigation of any \sw images that are
provided without world coordinate system (WCS) information will be
necessarily limited but the \sw Zooniverse \Letters will contribute to the collective knowledge and understanding of the presented system(s). Zooniverse \Letters are citable
objects, and in some cases will appear listed on ADS\footnote{To be confirmed.}. 


As \sw classifications are a community wide activity, it is impossible to attribute the discovery of a candidate to a single community member, or group  of members. Therefore, in recognition of their contribution to the \sw project, \sw community members will be listed (on their approval) on the \sw members web page. This web page will be linked in every ensuing \sw publication. If a community member makes a significant contribution e.g.\,in the further investigation of a lens candidate with modelling tools, they are strongly encouraged to write Zooniverse \Letters. This may lead to them being invited to join journal publications by the lead author of a ``discovery'' publication. 

%%%%%%%%%%%%%%%%%%%%%%%%%%%%%%%%%%%%%%%%%%%%%%%%%%%%%%%%%%%%%%%%%%%%%%%%%%%%%%

\section{Summary of Publication Guidelines}
\label{sec:publ}
Any \sw series and \sw enabled discovery papers should include
\begin{itemize}
\item citations to the \sw system papers, \sw I \& II
\item a small number of \sw authors proposed by the \sw PIs (see \ref{sec:papers}).
\item the collaboration should be acknowledged in the acknowledgement section with a link to the collaboration membership page (URL: to be confirmed when available).
\end{itemize}


%%%%%%%%%%%%%%%%%%%%%%%%%%%%%%%%%%%%%%%%%%%%%%%%%%%%%%%%%%%%%%%%%%%%%%%%%%%%%%

\end{document}

%%%%%%%%%%%%%%%%%%%%%%%%%%%%%%%%%%%%%%%%%%%%%%%%%%%%%%%%%%%%%%%%%%%%%%%%%%%%%%

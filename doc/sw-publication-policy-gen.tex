%%%%%%%%%%%%%%%%%%%%%%%%%%%%%%%%%%%%%%%%%%%%%%%%%%%%%%%%%%%%%%%%%%%%%%%%
%
% Space Warps Guidelines & Policy
%
%%%%%%%%%%%%%%%%%%%%%%%%%%%%%%%%%%%%%%%%%%%%%%%%%%%%%%%%%%%%%%%%%%%%%%%%

\documentclass[a4paper]{article}

% Packages:
\usepackage{xspace}
\usepackage{verbatim}
\usepackage{hyperref}

% Macros:
% JOURNALS
\newcommand{\apj}{ApJ}
\newcommand{\apjl}{ApJL}
\newcommand{\apjs}{ApJS}
\newcommand{\mnras}{MNRAS}
\newcommand{\apss}{Ap \& SS}
\newcommand{\aap}{A\&A}
\newcommand{\aj}{AJ}
\newcommand{\prd}{Phys. Rev. D}
\newcommand{\nat}{Nature}
\newcommand{\araa}{ARA\&A}
\newcommand{\jgr}{J. Geophys. Res.}
\newcommand{\pasp}{PASP}

% MISC
\newcommand{\etal}{et~al.~}
\newcommand{\eg}{{\it e.g.\ }}
\newcommand{\ie}{{\it i.e.\ }}
\newcommand{\etc}{{\it etc.\ }}

\newcommand{\be}{\begin{equation}}
\newcommand{\ee}{\end{equation}}
\newcommand{\bea}{\begin{eqnarray}}
\newcommand{\eea}{\end{eqnarray}}


% CROSS-REFERENCING
\def\Sref#1{Section~\ref{#1}\xspace}
\def\Fref#1{Figure~\ref{#1}\xspace}
\def\Tref#1{Table~\ref{#1}\xspace}
\def\Eref#1{Equation~\ref{#1}\xspace}
\def\Aref#1{Appendix~\ref{#1}\xspace}

% UNITS
\newcommand{\kms}{\ifmmode  \,\rm km\,s^{-1} \else $\,\rm km\,s^{-1}  $ \fi }
\newcommand{\kpc}{\ifmmode  {\rm kpc}  \else ${\rm  kpc}$ \fi  }  
\newcommand{\pc}{\ifmmode  {\rm pc}  \else ${\rm pc}$ \fi  }  
\newcommand{\Msun}{\ifmmode {\rm M_{\odot}} \else ${\rm M_{\odot}}$ \fi} 
\newcommand{\Zsun}{\ifmmode {\rm Z_{\odot}} \else ${\rm Z_{\odot}}$ \fi} 
\newcommand{\yr}{\ifmmode yr^{-1} \else $yr^{-1}$ \fi} 
\newcommand{\hMsun}{\ifmmode h^{-1}\,\rm M_{\odot} \else $h^{-1}\,\rm M_{\odot}$ \fi}

% COSMOLOGY
\newcommand{\LCDM}{$\Lambda{\rm CDM}$}
\newcommand{\MS}{Millennium Simulation\xspace}

% LENSING
\def\zd{z_{\rm d}}
\def\zs{z_{\rm s}}
\def\Dd{D_{\rm d}}
\def\Ds{D_{\rm s}}
\def\Dt{D_{\Delta t}}
\def\Dds{D_{\rm ds}}
\def\Sigmacrit{\Sigma_{\rm crit}}
\def\REin{R_{\rm Ein}}
\def\MEin{M_{\rm Ein}}

% SOFTWARE/HARDWARE
\def\sw{{\small\sc Space\,Warps}\xspace}
\def\SW{{\sc Space\,Warps}\xspace}
\def\Talk{{\small\sc Talk}\xspace}
\def\Letters{{\small\sc Letters}\xspace}
\def\Letter{{\small\sc Letter}\xspace}
\def\Dashboard{{\small\sc Dashboard}\xspace}
\def\cfhtls{{\it CFHTLS}\xspace}
\def\python{{\sc python}\xspace}

% TABLES:
\newcommand\nodata{ ~$\cdots$~ }%

% PROBABILITY THEORY
\def\pr{{\rm Pr}}
\def\data{{\mathbf{d}}}
\def\datap{{\mathbf{d}^{\rm p}}}
\def\datai{d_i}
\def\datapi{d^{\rm p}_i}
\def\LENS{{\rm LENS}}
\def\saidLENS{{\rm ``LENS"}}
\def\NOT{{\rm NOT}}
\def\saidNOT{{\rm ``NOT"}}

% AGENT BUREAUCRACY
\def\effort{N_{\rm C}}
\def\experience{N_{\rm T}}
\def\skill{\langle I \rangle}
\def\contribution{$\sum_k \skill_k$}
\def\information{\delta I}

% COMMENTING
\usepackage[usenames]{color}
\newcommand{\question}[2]{\textcolor{red}{\bf Question from #1: #2}}
\newcommand{\flag}[2]{\textcolor{blue}{\bf Comment from #1: #2}}
\newcommand{\new}[1]{{\bf #1}}

% RESULTS
\def\Ncollaboration{XXX}


%%%%%%%%%%%%%%%%%%%%%%%%%%%%%%%%%%%%%%%%%%%%%%%%%%%%%%%%%%%%%%%%%%%%%%%%

\begin{document}
             
\title{Guidelines and Policy for Surveys using \SW}
\author{Phil Marshall, Aprajita Verma \& Anupreeta More}
\date{April 15, 2013}
\maketitle

%%%%%%%%%%%%%%%%%%%%%%%%%%%%%%%%%%%%%%%%%%%%%%%%%%%%%%%%%%%%%%%%%%%%%%%%%%%%%%

\begin{abstract} 
\noindent \sw was conceived as a strong lens
discovery {\it service}, that any survey science team can make use of
(within the logistical constraints). In this document we discuss the relationship
between the survey science teams and the \sw collaboration. 
This document includes guidelines for data provision, the data policy and for publications of \sw-enabled discoveries.
\end{abstract}

\setcounter{footnote}{0}

%%%%%%%%%%%%%%%%%%%%%%%%%%%%%%%%%%%%%%%%%%%%%%%%%%%%%%%%%%%%%%%%%%%%%%%%%%%%%%

\section{Introduction to \sw}

The \sw website is designed to motivate and  enable tens of thousands of
people to perform the scientific tasks of strong gravitational lens
identification and classification.  \sw comprises a primary classification interface (http://spacewarps.org) that allows citizen scientists to inspect images and mark those that contain potential gravitational lenses. In addition, these volunteers are shown simulated lenses (appropriate for the survey being conducted) as an interactive training tool that also allows us to establish the citizens' likelihood of identifying potential lenses and blank fields using the \sw Analysis Pipeline (SWAP). 

A key part of the \sw project is the ability for all members of the \sw collaboration (see Section \ref{sec:members}) to communicate with each other and discuss interesting candidates through \Talk\footnote{\Talk (found at \texttt{http://talk.spacewarps.org}) is a forum for \sw collaborators to discuss individual sources and the \sw project in general. Bugs can also be reported. This is the direct portal for communication with the \sw community.}).



%%%%%%%%%%%%%%%%%%%%%%%%%%%%%%%%%%%%%%%%%%%%%%%%%%%%%%%%%%%%%%%%%%%%%%%%%%%%%%


\section{Collaboration Membership Types}
\label{sec:members}

The \sw collaboration comprises the \sw science team, \sw development team, and all the volunteers who have 
logged in\footnote{with a Zooniverse ID. It is possible to classify anonymously, and we count those classification but as we don't know who they are, we are unable to collaborate with them directly.} and contributed classifications to the project. 

An expanded definition of the \sw membership is given in the document ``\sw Publication Policy: CFHT-LS''.  In this document we discuss the relationship between the \sw collaboration and future Survey Science Teams (SST) who wish to use the \sw interface to investigate their data. The SSTs will liaise with the \sw science and development teams to ingest their survey data into the \sw interface. \sw was established with future surveys in mind and therefore there may be some overlap between current \sw members and the future SSTs.


In the remaining sections we look at discovery publications that might
arise from future \sw projects, and suggest reasonable guidelines for
deciding on their authorship. First, however, we remind ourselves about data
access via \sw.

%%%%%%%%%%%%%%%%%%%%%%%%%%%%%%%%%%%%%%%%%%%%%%%%%%%%%%%%%%%%%%%%%%%%%%%%%%%%%%


\section{Survey Data Policy}
\label{sec:data}

\sw is a public website. All images displayed there are by definition in the
public domain, and so must be expected to be downloaded, copied and
redistributed by any \sw user. This is a \textbf{good thing}: images of the sky taken
with publicly-funded telescopes belong to everyone. 

However, some surveys come with their own proprietary access policy. It is the responsibility
of the SSTs to provide images to \sw in a way that is consistent
with their own data access rules. There are two things that SSTs can
do to in order to respect any proprietary period that they have imposed on
themselves:
\begin{enumerate}

\item Add a ``LICENSE'' keyword to the FITS and PNG image headers, explaining
what the rules for redistribution of these images are. This will almost
certainly be ignored, but it would be a nice reminder that nothing comes free
of either cost or responsibility. Other keywords could also be included as
well: data provenance is important, and links to useful survey webpages would
be most welcome!

\item Remove all WCS information from all images provided. The \sw interface ``dashboards'' do allow FITS and JPG files to be downloaded. If images from a proprietary dataset being made available is a concern for an SST, they should remove the WCS. This also has the consequence that it minimises the potential ``scooping'' of  lens candidates by anyone other than the SST. 
In practice, the objects contained in the \sw images will be too faint
for anyone without access to a large telescope to observe, and may also be
absent from any public catalogs, so the opportunities for follow-up will be
quite limited. Images
with field of view less than about 3~arcminutes in diameter are not solvable
by \texttt{astrometry.net}. \textbf{Please note}, however, that we cannot exclude the possibility that object co-ordinates are posted in \Talk. For example, co-ordinates of previously published lens candidates or recognisable fields from CFHT-LS have been posted in \Talk by a very small number of citizens.

\end{enumerate}

%%%%%%%%%%%%%%%%%%%%%%%%%%%%%%%%%%%%%%%%%%%%%%%%%%%%%%%%%%%%%%%%%%%%%%%%%%%%%%

\section{Data Preparation and Publications for Future Survey Teams}
\label{sec:series}

\subsection{Future Project Proposals}
Any future \sw projects are requested to liaise with the \sw PIs regarding their project. We ask that potential project teams first approach the \sw PIs (Phil Marshall, Anupreeta More, Aprajita Verma, collectively contacted via \href{mailto:spacewarpspi@googlegroups.com}{spacewarpspi@googlegroups.com}) with a short description of their project that includes the following information.

\begin{itemize}
\item Survey
\item Bands to be included
\item Total number of images to be viewed (approx.)
\item Targeted survey or blind field search \footnote{Targeted: inspection of a list of candidates; Blind: inspection of tiles of a survey area}
\item Brief outline of the overall goal(s)
\item Proprietary data issues 
\end{itemize}

The \sw PIs will then arrange a telecon with the SSTs to discuss their project further.

The original \sw papers, by definition, provide a complete description of the system, and its results from the CFHT-LS survey. The science team of
any other survey (such as DES, RCS, KIDS, PS1, HSC \etc) should be able
to read these papers before designing their own \sw project, and cite them as justification for some of their experimental decisions. As the \sw paper series get published, links to the papers will be given in this draft. Prior to publications, draft papers can also be shared on request by the \sw PIs (\href{mailto:spacewarpspi@googlegroups.com}{spacewarpspi@googlegroups.com})

\subsection{Collaborating with \sw Science Team}

Implementing a new project based on different survey data will require support from the \sw Science Team, in three respects: 

\begin{enumerate}

\item In {\bf preparing data} for the site, and uploading it onto the
Zooniverse servers. There is a reasonable amount of work in preparing data and images by the SSTs for delivery to \sw. The \sw team will provide assistance to SSTs in data preparation. The benefits of using the \sw service will far outweigh the time needed to prepare data.  The \sw Science Team can advise on suitable Training
Subjects and image display settings, and also on formatting the data
ready for ingestion and display.

\item In {\bf reconfiguring the site itself} ready for the new data. The
SST will need to include new Spotter's Guide images and text,
modified tutorial content, and additional survey-specific site content;
the \sw Science Team can help with all of this.

\item In {\bf maintaining the site} so that it continues to function
correctly. This will involve technical support from the Zooniverse and \sw science teams,
but will also require the continued engagement of the wider
collaboration by the SST in \Talk. Since this social filter is
a key part of the sample refinement, the latter may involve a
significant time commitment.

\end{enumerate}

\subsection{Resulting Publications}

\subsubsection{Journal Papers}
\label{sec:papers}
In return for the work outlined in Section \ref{sec:series} in setting up the new surveys, the \sw PIs will nominate a
small number of core \sw collaboration members as authors on \textbf{papers that
present discovery of a lens or sample of lenses enabled by the \sw system}. A ``discovery'' is defined by the announcement of the source(s) to the astronomical community in a journal paper. We would expect the \sw author list to be a small number (less than ten), primarily the Principal Investigators and a few others working on the ingestion of the new data. All post-discovery follow-up papers would of course be exempt from this: the idea is that \sw is purely a gravitational lens discovery service, but not one that runs by magic.

We request that authors of \sw-enabled discovery papers should circulate drafts to the list of \sw authors and allow a minimum of two weeks to give time for comments. We also request that any papers based on \sw enabled discoveries should cite the \sw system papers. This policy document will be updated with the correct citations to the \sw system papers when available.

\sw runs an analysis pipeline (SWAP) that will be tailored to the science goals of each new survey. Any large changes to the analysis code will be documented  by the \sw Science Team. The \sw Science Team reserve the right to update the \sw methodology papers.




%%%%%%%%%%%%%%%%%%%%%%%%%%%%%%%%%%%%%%%%%%%%%%%%%%%%%%%%%%%%%%%%%%%%%%%%%%%%%%

\subsubsection{Zooniverse \Letters}
\label{sec:comm}

Any \sw collaboration member, simply by virtue of their Zooniverse
registration, may write a Zooniverse
Letter\footnote{\texttt{http://letters.zooniverse.org}} describing their
investigation of any lens candidate they find in \sw. This, along with
posts in  \Talk, is the primary means by which we expect collaboration
members will communicate their findings to the rest of the
astronomical community. The investigation of any \sw images that are
provided without world coordinate system (WCS) information will be
necessarily limited but the \sw Zooniverse \Letters will contribute to the collective knowledge and understanding of the presented system(s). Zooniverse \Letters are citable
objects, and in some cases will appear listed on ADS\footnote{To be confirmed.}. 


As \sw classifications are a community wide activity, it is impossible to attribute the discovery of a candidate to a single community member, or group  of members. Therefore, in recognition of their contribution to the \sw project, \sw community members will be listed (on their approval) on the \sw members web page. This web page will be linked in every ensuing \sw publication. If a community member makes a significant contribution e.g.\,in the further investigation of a lens candidate with modelling tools, they are strongly encouraged to write Zooniverse \Letters. This may lead to them being invited to join journal publications by the lead author of a ``discovery'' publication. 

%%%%%%%%%%%%%%%%%%%%%%%%%%%%%%%%%%%%%%%%%%%%%%%%%%%%%%%%%%%%%%%%%%%%%%%%%%%%%%

\section{Summary of Publication Guidelines}
\label{sec:publ}
Any \sw series and \sw enabled discovery papers should include
\begin{itemize}
\item citations to the \sw system papers, \sw I \& II
\item a small number of \sw authors proposed by the \sw PIs (see \ref{sec:papers}).
\item the collaboration should be acknowledged in the acknowledgement section with a link to the collaboration membership page (URL: to be confirmed when available).
\end{itemize}


%%%%%%%%%%%%%%%%%%%%%%%%%%%%%%%%%%%%%%%%%%%%%%%%%%%%%%%%%%%%%%%%%%%%%%%%%%%%%%

\end{document}

%%%%%%%%%%%%%%%%%%%%%%%%%%%%%%%%%%%%%%%%%%%%%%%%%%%%%%%%%%%%%%%%%%%%%%%%%%%%%%
